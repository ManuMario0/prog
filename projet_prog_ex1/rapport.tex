\documentclass[12pt,a4paper]{article}
\usepackage[utf8]{inputenc}
\usepackage[french]{babel}
\usepackage[T1]{fontenc}
\usepackage{amsmath}
\usepackage{amsfonts}
\usepackage{amssymb}
\usepackage{minted}
%\usepackage{xcolor}
\usepackage{caption}
\usepackage{geometry}
\newenvironment{longlisting}{\captionsetup{type=listing, width=\linewidth}}{}
\author{Emmanuel MÉRA}
\title{Structure de liste en Ocaml}
\begin{document}
\maketitle
\section{Introduction}
	Bonjour, je suis une introduction.
\section{Code source}
	\begin{longlisting}
	\begin{minted}[breaklines, autogobble, linenos, frame = lines, fontsize=\footnotesize, baselinestretch=.9]{Ocaml}
type 'a my_list =
        | Nil
        | Cons of 'a * 'a my_list
;;

let rec string_of_list f l =
        match l with
        | Nil -> ""
        | Cons(a, b) -> (f a)^(string_of_list f b)
;;

let hd l =
        match l with
        | Nil -> None
        | Cons (a, _) -> Some a
;;

let tl l =
        match l with
        | Nil -> None
        | Cons (_, b) -> Some b
;;

let rec length l =
        match l with
        | Nil -> 0
        | Cons (_, b) -> 1 + (length b)
;;

let rec map f l =
        match l with
        | Nil -> Nil
        | Cons (a, b) -> Cons(f a, map f b)
;;
	\end{minted}
	\end{longlisting}
	
\section{Exemples}
	\begin{longlisting}
	\caption{Code source}
	\begin{minted}[breaklines, autogobble, linenos, frame = lines, fontsize=\footnotesize, baselinestretch=.9]{Ocaml}
open My_list;;

let string_of_list str_fun l =
  let rec string_content = function
    | []  -> ""
    | [x]  -> (str_fun x)
    | x::l -> (str_fun x) ^ ", " ^ (string_content l)
  in "[" ^ (string_content l) ^ "]" in

let string_of_nat_list = string_of_list string_of_int in
let string_of_string_list = string_of_list (fun x -> x) in

let empty = [] in
let one = ["a"] in
let lst = [1; 3; 6; 10; 15; 21; 28; 36; 45; 55] in

let test_hd () =
  Printf.printf "Tête de %s : %s.\n" (string_of_string_list one) (List.hd one);
  Printf.printf "Tête de %s : %d.\n\n" (string_of_nat_list lst) (List.hd lst)

in let test_tl () =
  Printf.printf "Queue de %s : %s.\n" (string_of_string_list one) (string_of_string_list (List.tl one));
  Printf.printf "Queue de %s : %s.\n\n" (string_of_nat_list lst) (string_of_nat_list (List.tl lst))

in let test_length () =
  Printf.printf "Taille de %s : %d.\n" (string_of_string_list one) (List.length one);
  Printf.printf "Taille de %s : %d.\n" (string_of_nat_list lst) (List.length lst);
  Printf.printf "Taille de %s : %d.\n\n" (string_of_string_list empty) (List.length empty)

in let test_map ()=
  Printf.printf "Map de (x -> xx) sur %s : %s.\n" (string_of_string_list one) (string_of_string_list (List.map (fun s -> s ^ s) one));
  Printf.printf "Map de (x -> 2x) sur %s : %s.\n" (string_of_nat_list lst) (string_of_nat_list (List.map (fun n -> 2 * n) lst));
  Printf.printf "Map de (x -> 2x) sur %s : %s.\n\n" (string_of_nat_list empty) (string_of_nat_list (List.map (fun n -> 2 * n) empty));

in test_hd(); test_tl(); test_length(); test_map()
	\end{minted}
	\end{longlisting}
	
	\begin{longlisting}
	\caption{Output}
	\begin{minted}[breaklines, autogobble, linenos, frame = lines, fontsize=\footnotesize, baselinestretch=.9]{bash}
Tête de [a] : a.
Tête de [1, 3, 6, 10, 15, 21, 28, 36, 45, 55] : 1.

Queue de [a] : [].
Queue de [1, 3, 6, 10, 15, 21, 28, 36, 45, 55] : [3, 6, 10, 15, 21, 28, 36, 45, 55].

Taille de [a] : 1.
Taille de [1, 3, 6, 10, 15, 21, 28, 36, 45, 55] : 10.
Taille de [] : 0.

Map de (x -> xx) sur [a] : [aa].
Map de (x -> 2x) sur [1, 3, 6, 10, 15, 21, 28, 36, 45, 55] : [2, 6, 12, 20, 30, 42, 56, 72, 90, 110].
Map de (x -> 2x) sur [] : [].
	\end{minted}
	\end{longlisting}
	\section{Conclusion}
	Bonjour, je suis une conclusion.
\end{document}
